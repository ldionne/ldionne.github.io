\documentclass[10pt]{moderncv}
\usepackage[utf8]{inputenc}
\usepackage[T1]{fontenc}
\usepackage[super]{nth}
\usepackage[hscale=0.8, vscale=0.825]{geometry} % required to fit on one page. should be scale=0.8
\nopagenumbers

% Links do not appear colored if we only use \href.
\newcommand{\myhref}[3][blue]{\href{#2}{\color{#1}{#3}}}

\moderncvtheme[blue]{classic}
\firstname{Louis}
\familyname{Dionne}
\email{ldionne.2@gmail.com}
\homepage{ldionne.com}


\begin{document}
\maketitle


% \cventry{} % year
%         {} % job title
%         {} % employer
%         {} % city
%         {} % general description
%         {} % details

%%%%%%%%%%%%%%%%%%%%%%%%%%%%%%%%%%%%%%%%%%%%%%%%%%%%%%%%%%%%%%%%%%%%%%%%%%%%%%

\section{Education}
\cventry{Jan 2013 -- present}{B.Sc Mathematics}{Université Laval}{Québec}{}{}

\section{Experience}
\cventry{2014 - present}{Various C++ consulting work}{}{}{}{}

\cventry{2014 - present}{GSoC student with Boost}{\myhref{http://www.google-melange.com/gsoc/homepage/google/gsoc2014}{Google Summer of Code}}{}{}{
    Work on \myhref{http://github.com/ldionne/hana}{Boost.Hana} during the summers
    of 2014 and 2015. I also received a grant by the \myhref{https://goo.gl/JVj1me}
    {Boost Steering Committee} to continue my GSoC work during the winter of 2015.
}

\cventry{2013}{Student volunteer}{\myhref{http://www.cppnow.org}{C++Now}}{Aspen}{}{
    Help the conference staff with logistics, recording talks and other tasks
}

\cventry{2012}{Intern}{\myhref{http://www.coveo.com}{Coveo Solutions}}{Québec}{}{
    \begin{itemize}
        \item Conception and implementation of a deadlock detection system for
              internal use
        \item Presentations on C++ techniques and idioms to co-workers:
            \begin{itemize}
                \item The Boost.ConceptCheck library and associated TMP techniques
                \item C++11 rvalue references
            \end{itemize}
    \end{itemize}
}

\cventry{2010}{Intern}{Department of Electrical Engineering of Université Laval}{Québec}{}{
   Organization of the \nth{10} international conference on
   \myhref{http://qirt.org}{Quantitative Infrared Thermography (QIRT)}
}

%%%%%%%%%%%%%%%%%%%%%%%%%%%%%%%%%%%%%%%%%%%%%%%%%%%%%%%%%%%%%%%%%%%%%%%%%%%%%%

\section{Talks}
\cventry{2015}
        {Metaprogramming: a paradigm shift (\myhref{http://ldionne.github.io/hana-cppcon-2015}{slides})}
        {\myhref{http://cppcon.org}{CppCon}}{Seattle}{}{}{}

\cventry{2015}
        {Metaprogramming: a paradigm shift (\myhref{http://ldionne.github.io/hana-cppnow-2015}{slides})}
        {\myhref{http://www.cppnow.org}{C++Now}}{Aspen}{}
        {Awards for the best presentation and the most inspiring presentation}

\cventry{2014}
        {Metaprogramming in C++14 (\myhref{http://ldionne.github.io/hana-opencode-xxii}{french only slides})}
        {\myhref{http://www.opencode.ca}{OpenCode XXII}}{Québec}{}{}

\cventry{2014}
        {Hana: Expressive metaprogramming (\myhref{http://ldionne.github.io/hana-cppcon-2014/}{slides}/\myhref{https://www.youtube.com/watch?v=L2SktfaJPuU}{video})}
        {\myhref{http://cppcon.org}{CppCon}}{Seattle}{}{}

\cventry{2014}
        {Towards painless metaprogramming (\myhref{http://ldionne.github.io/mpl11-cppnow-2014}{slides}/\myhref{https://www.youtube.com/watch?v=8c0aWLuEO0Y}{video})}
        {\myhref{http://www.cppnow.org}{C++Now}}{Aspen}{}{}

\cventry{2013}
        {A system for resource deadlock prevention (\myhref{http://ldionne.github.io/d2-cppnow-2013}{slides}/\myhref{https://www.youtube.com/watch?v=Re67U4zAN-M}{video})}
        {\myhref{http://www.cppnow.org}{C++Now}}{Aspen}{}{}

\cventry{2013}
        {Deadlock detection with d2 (\myhref{http://ldionne.github.io/d2-opencode12}{slides})}
        {\myhref{http://www.opencode.ca}{OpenCode XII}}{Québec}{}{}

\cventry{2013}
        {Concept based overloading in C++ (\myhref{http://docs.google.com/presentation/d/1HpjEz6dJauNoBxMGWzaNuOOoiVwqgISTNWNTwuOo-_8/edit?usp=sharing}{slides})}
        {\myhref{http://www.opencode.ca}{OpenCode IX}}{Québec}{}{}


%%%%%%%%%%%%%%%%%%%%%%%%%%%%%%%%%%%%%%%%%%%%%%%%%%%%%%%%%%%%%%%%%%%%%%%%%%%%%%

\section{Personal Projects}
\newcommand{\PersonalProject}[4]{
    \cvline{\myhref{#2}{#1}}{
        \textbf{#3}\newline
        #4
    }
}

\PersonalProject{Boost.Hana}{http://github.com/ldionne/hana}
{A new C++14 Boost library for heterogeneous computation}{
    Design and implementation of a library to manipulate heterogeneous
    sequences at compile-time and at runtime. The library introduces a
    new paradigm for expressing meta-computations with regular functions
    instead of quirky template tricks, achieving a new level of expressiveness.
    The library is implemented using the most recent advances in C++14
    metaprogramming, making long compile-times (mostly) a thing of the past.}

\PersonalProject{mpl11}{http://github.com/ldionne/mpl11}
{Conception and implementation of a C++11 replacement for the Boost.MPL}{
    Reimplemented the functionality of the Boost.MPL library using new
    template metaprogramming techniques made possible by C++11. Redesigned
    the API of the library using ideas from Haskell to make it more powerful,
    easier to use and to extend.}

\PersonalProject{d2}{http://github.com/ldionne/d2}
{Conception and implementation of a deadlock detection system in C++}{
    Detects deadlocks that would have happened under different thread scheduling
    conditions by performing intrusive dynamic analysis on a non-deadlocking run
    of a program. Additionally, provides satistics about lock and thread usage.}

\PersonalProject{joy}{http://github.com/ldionne/joy}
{Implementation of a preprocessor metaprogramming library}{
    Implemented associative sequences and other utilities for preprocessor
    metaprogramming on top of the
    \myhref{http://sourceforge.net/projects/chaos-pp}{Chaos preprocessor library}.}

\PersonalProject{nstl}{http://github.com/ldionne/nstl}
{Conception and implementation of a generic algorithm library in pure C}{
    Implemented a basic name mangling system and ``preprocessor-based classes''
    using PMP techniques. Using these facilities, implemented a subset of the
    C++ standard library algorithms. The result is a collection of generic
    algorithms instantiable and usable from pure C without sacrificing type
    safety or performance by using traditional techniques like pointers to void.}

\PersonalProject{duck}{http://github.com/ldionne/duck}
{Implementation of a minimal concept-based overloading library}{
    Implemented a subset of Boost.ConceptCheck's concepts as metafunctions,
    which allows overloading based on the modeling of a concept by a type.}

\PersonalProject{cisp}{http://github.com/ldionne/cisp}
{Implementation of a minimalist object system with the preprocessor}{
    Created a system to manipulate complex preprocessor objects using
    associative sequences imbued with object semantics.}

\PersonalProject{nstl-lang}{http://github.com/ldionne/nstl-lang}
{Implementation of a translator for a toy language in Python}{
    Implemented basic parsing, semantic analysis and code generation to C.}

\cvline{}
{\textbf{Contributions to other projects}
\begin{itemize}
    \item Contribution of the \myhref{http://www.boost.org/doc/libs/release/libs/graph/doc/hawick_circuits.html}{hawick\_circuits} algorithm to Boost.Graph
    \item Occasional patches to Boost (Spirit, Graph, Archive, MPL and others)
    \item Active on the \myhref{http://news.gmane.org/gmane.comp.lib.boost.devel}{Boost.Dev} mailing list
    \item CMake port of the \myhref{http://github.com/lemire/FastPFor}{FastPFor}
          integer compression library's build system
\end{itemize}}

\end{document}
