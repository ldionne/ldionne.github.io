\documentclass[10pt]{moderncv}
\usepackage[utf8]{inputenc}
\usepackage[T1]{fontenc}
\usepackage[super]{nth}
\usepackage[hscale=0.8, vscale=0.825]{geometry} % required to fit on one page. should be scale=0.8
\nopagenumbers

% Links do not appear colored if we only use \href.
\newcommand{\myhref}[3][blue]{\href{#2}{\color{#1}{#3}}}

\moderncvtheme[blue]{classic}
\firstname{Louis}
\familyname{Dionne}
\email{ldionne.2@gmail.com}
\homepage{ldionne.com}

% Make sure the dates in the left column fit on one line
\setlength{\hintscolumnwidth}{3.24cm}

\begin{document}
\maketitle


% \cventry{} % year
%         {} % job title
%         {} % employer
%         {} % city
%         {} % general description
%         {} % details

%%%%%%%%%%%%%%%%%%%%%%%%%%%%%%%%%%%%%%%%%%%%%%%%%%%%%%%%%%%%%%%%%%%%%%%%%%%%%%

\section{Education}
\cventry{Jan 2013 -- Dec 2015}{B.Sc. Mathematics}{Université Laval}{Québec}{}{}
\cventry{Sep 2011 -- May 2012}{B.Sc. Software Engineering (not completed)}{Université Laval}{Québec}{}{}

\section{Experience}
\cventry{May 2017 -- present}{Member of the Boost Steering Committee}{\myhref{http://www.boost.org}{Boost.org}}{}{}{
    Participate in technical decisions impacting the future of the Boost community,
    but also day-to-day procedural and policy-related issues.
}

\cventry{Dec 2016 -- present}{Amazon Representative on the C++ Standards Committee}{}{}{}{}

\cventry{Jun 2016 -- present}{Software Development Engineer}{\myhref{http://a9.com}{A9.com (an Amazon company)}}{Palo Alto}{}{
    Member of the search infrastructure team powering Amazon's search engine.
}

\cventry{Dec 2014 -- Jun 2016}{C++ consulting}{}{}{(finance, embedded systems)}{
    Development of C++ libraries to retain a high level of abstraction
    in applications where both performance and correctness matter. Also
    some refactoring of existing systems to add new features and/or
    improve performance.
}

\cventry{2014 -- 2017}{Development of \myhref{http://github.com/boostorg/hana}{Hana}}{}{}
                      {a \myhref{http://boost.org}{Boost} library for C++ metaprogramming}{
    Design and implementation of a library to manipulate heterogeneous
    sequences at compile-time and at runtime. The library introduces a
    new paradigm for expressing meta-computations allowing a very high
    level of expressiveness with little to no performance penalty.
}

\cventry{2014 -- 2015}{GSoC student with Boost}{\myhref{http://www.google-melange.com/gsoc/homepage/google/gsoc2014}{Google Summer of Code}}{}{}{
    Work on \myhref{http://github.com/ldionne/hana}{Boost.Hana} during the summers of
    2014 and 2015 as part of the Google Summer of Code program. I also received a grant
    from the \myhref{http://boost.org}{Boost} Steering Committee to continue working
    during the winter of 2015, which had never been done before for a GSoC student.
}

% \cventry{2013}{Student volunteer}{\myhref{http://www.cppnow.org}{C++Now}}{Aspen}{}{
%     Help the conference staff with logistics, recording talks and other tasks
% }

\cventry{Sep 2012 -- Dec 2012}{Software Developer}{\myhref{http://www.coveo.com}{Coveo Solutions}}{Québec}{}{
    Work on a MIME parser in C++. Resigned to pursue a degree in mathematics.
}

\cventry{May 2012 -- Aug 2012}{Intern}{\myhref{http://www.coveo.com}{Coveo Solutions}}{Québec}{}{
    \begin{itemize}
        \item Conception and implementation of a deadlock detection system for
              internal use
        \item Presentations on C++ techniques and idioms to co-workers:
            \begin{itemize}
                \item The \myhref{http://www.boost.org/doc/libs/release/libs/concept_check/concept_check.htm}{Boost.ConceptCheck}
                      library and associated template metaprogramming techniques
                \item C++11 rvalue references
            \end{itemize}
    \end{itemize}
}

% \cventry{2010}{Intern}{Department of Electrical Engineering of Université Laval}{Québec}{}{
%   Organization of the \nth{10} international conference on
%   \myhref{http://qirt.org}{Quantitative Infrared Thermography (QIRT)}
% }

%%%%%%%%%%%%%%%%%%%%%%%%%%%%%%%%%%%%%%%%%%%%%%%%%%%%%%%%%%%%%%%%%%%%%%%%%%%%%%

\section{Selected talks (\myhref{http://ldionne.github.io/talks}{full list})}
\cventry{2017}
        {Runtime Polymorphism: Back to the Basics (\myhref{http://ldionne.com/cppcon-2017-runtime-polymorphism}{slides}/\myhref{https://youtu.be/gVGtNFg4ay0}{video})}
        {\myhref{http://cppcon.org}{CppCon}}{Bellevue}{}{}

\cventry{2016}
        {Closing keynote on metaprogramming (\myhref{http://ldionne.github.io/meetingcpp-2016}{slides}/\myhref{https://youtu.be/X_p9X5RzBJE}{video})}
        {\myhref{https://meetingcpp.com}{Meeting C++}}{Berlin}{}
        {Was voted the best presentation by attendees}

\cventry{2015}
        {Metaprogramming: a paradigm shift (\myhref{http://ldionne.github.io/hana-cppnow-2015}{slides}/\myhref{http://youtu.be/Z2ABRaQiFHs}{video})}
        {\myhref{http://www.cppnow.org}{C++Now}}{Aspen}{}
        {Awards for the best presentation and the most inspiring presentation}

\cventry{2014}
        {Hana: Expressive metaprogramming (\myhref{http://ldionne.github.io/hana-cppcon-2014/}{slides}/\myhref{https://www.youtube.com/watch?v=L2SktfaJPuU}{video})}
        {\myhref{http://cppcon.org}{CppCon}}{Bellevue}{}{}

\cventry{2013}
        {A system for resource deadlock prevention (\myhref{http://ldionne.github.io/d2-cppnow-2013}{slides}/\myhref{https://www.youtube.com/watch?v=Re67U4zAN-M}{video})}
        {\myhref{http://www.cppnow.org}{C++Now}}{Aspen}{}{}

%%%%%%%%%%%%%%%%%%%%%%%%%%%%%%%%%%%%%%%%%%%%%%%%%%%%%%%%%%%%%%%%%%%%%%%%%%%%%%

\section{Personal Projects}
\newcommand{\PersonalProject}[4]{
    \cvline{\myhref{#2}{#1}}{
        \textbf{#3}\newline
        #4
    }
}

\PersonalProject{Metabench}{http://ldionne.github.io/metabench}
{A simple framework for doing compile-time benchmarks}{
    Implemented a self-contained CMake module to perform compile-time benchmarks
    of C++ metaprograms. Such benchmarks are very useful when writing a
    metaprogramming library, where the performance of the library must be
    measured in terms of compilation time. The module works by having the user
    write \myhref{http://en.wikipedia.org/wiki/ERuby}{ERB} templates that are
    then used to generate C++ programs. The C++ programs are compiled and various
    metrics such as compilation time, link time and executable size are gathered.
    The module generates HTML5 charts to easily visualize the metrics.}

\PersonalProject{mpl11}{http://github.com/ldionne/mpl11}
{Conception and implementation of a C++11 replacement for the Boost.MPL}{
    Reimplemented the functionality of the Boost.MPL library using new
    template metaprogramming techniques made possible by C++11. Redesigned
    the API of the library using ideas from Haskell to make it more powerful,
    easier to use and to extend.}

\PersonalProject{d2}{http://github.com/ldionne/d2}
{Conception and implementation of a deadlock detection system in C++}{
    Detects deadlocks that would have happened under different thread scheduling
    conditions by performing intrusive dynamic analysis on a non-deadlocking run
    of a program. Additionally, provides statistics about lock and thread usage.}

\PersonalProject{joy}{http://github.com/ldionne/joy}
{Implementation of a preprocessor metaprogramming library}{
    Implemented associative sequences and other utilities for preprocessor
    metaprogramming on top of the
    \myhref{http://sourceforge.net/projects/chaos-pp}{Chaos preprocessor library}.}

\PersonalProject{nstl}{http://github.com/ldionne/nstl}
{Conception and implementation of a generic algorithm library in pure C}{
    Implemented a basic name mangling system and ``preprocessor-based classes''
    using PMP techniques. Using these facilities, implemented a subset of the
    C++ standard library algorithms. The result is a collection of generic
    algorithms instantiable and usable from pure C without sacrificing type
    safety or performance by using traditional techniques like pointers to void.}

\PersonalProject{duck}{http://github.com/ldionne/duck}
{Implementation of a minimal concept-based overloading library}{
    Implemented a subset of Boost.ConceptCheck's concepts as metafunctions,
    which allows overloading based on the modeling of a concept by a type.}

\PersonalProject{cisp}{http://github.com/ldionne/cisp}
{Implementation of a minimalist object system with the preprocessor}{
    Created a system to manipulate complex preprocessor objects using
    associative sequences imbued with object semantics.}

\PersonalProject{nstl-lang}{http://github.com/ldionne/nstl-lang}
{Implementation of a translator for a toy language in Python}{
    Implemented basic parsing, semantic analysis and code generation to C.}

\cvline{}
{\textbf{Contributions to other projects}
\begin{itemize}
    \item Contribution of the \myhref{http://www.boost.org/doc/libs/release/libs/graph/doc/hawick_circuits.html}{hawick\_circuits} algorithm to Boost.Graph
    \item Occasional patches to Boost (Spirit, Graph, Archive, MPL and others)
    \item Active on the \myhref{http://news.gmane.org/gmane.comp.lib.boost.devel}{Boost.Dev} mailing list
    \item CMake port of the \myhref{http://github.com/lemire/FastPFor}{FastPFor}
          integer compression library's build system
    \item Too frequent bug reports against the \myhref{https://goo.gl/y9pYWP}{Clang}
          and \myhref{https://goo.gl/IhKUrK}{GCC} compilers.
\end{itemize}}

\end{document}
